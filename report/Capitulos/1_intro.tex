\chapter{Introdução}

    O ser humano interage e necessita interagir com ondas a todo momento. Essas pertubações mecânicas ou eletromagnéticas fazem parte do nosso cotidiano desde o nosso acordar, pela luz (um tipo de onda eletromagnética) que é emitida/refletida até nós e que nos permite observar o mundo ao nosso redor pelos nossos olhos ou pelo som (composto por ondas mecânicas) emitido por carros, fábricas, choro de crianças ou um belo canto de pássaros que são captados por nossos ouvidos. Interagimos com as ondas até indiretamente, como o leitor agora interage, por exemplo, ou estando sentado em uma cadeira ou andando sobre um calçado ou se deslocando em algum meio de transporte, que provavelmente tem algum mineral adquirido através de estudos sismológicos, que utilizam massivamente os conceitos de ondulatória.
    
    Na área da Geologia (e também na Medicina, Física, construção civíl, por exemplo), aplicam-se, pelo menos, dois métodos para o estudo de fenômenos ondulatórios e de estruturas geológicas não homogêneas: o \textbf{método de diferenças finitas} e o \textbf{traçamento de raios}. Utiliza-se esses métodos (principalmente o segundo) na busca por minérios e pela história da formações de estruturas geológicas.
    
    Neste trabalho, objetivamos comparar esses dois métodos numéricos computacionais na simulação da propagação de ondas através de meios não-homogêneos. Tal comparação se dará nos aspectos da dificuldade de implementação dos métodos, seus custos, sua intuitividade e qual seria a melhor área para a aplicação de tais métodos.
    
    Para alcançar tal objetivo, o autor deste relatório e orientando dessa iniciação científica passa por uma série de estudos prévios (Capítulos \ref{cap:cientComp}, \ref{cap:EDOs}, \ref{cap:aproxPVI}, \ref{cap:ond} e \ref{cap:EDPs}) que dão base ao estudo realizado nesse trabalho, para enfim abordar os métodos a serem comparados (Capítulos \ref{cap:MDF}, \ref{cap:RT} e \ref{cap:CompMDFeRT}).
    
    Os principais códigos desenvolvidos durante o projeto, bem como esse próprio relatório, podem ser encontrados no repositório do autor (\url{https://github.com/MarceloFCandido/PIC}). Os códigos envolvidos diretamente na comparação dos métodos também se encontram no apêndice desse relatório.